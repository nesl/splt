\section{Related Work}
Related work to this work can be seen in two folds; a) Water flow rate measurement, b) Infrastructure monitoring regarding activity classification.
Water flow rate measurement has been of great interest in many of fields of study; chemical processes often require precise control over fluid flow rate, irrigation network needs to be monitored to avoid abuse of water[], and most of utility companies deploy intrusive water flow meters for billing purpose at most of houses and buildings. We can divide these into two categories; a) open channel water monitoring/discharged water monitoring from pipes and b) water flow rate estimate in closed pipes such as plumbing system in houses and industries.  
Large scale irrigation system requires system engineers to monitor water flow rate in open channels and closed channels. Hohn[] described a way to estimate discharged water flow rate from pipes by observing height and pipe diameter. [][][] provided some look-up tables that map manual observation to water flow rate in open channels. Most of them rely on manual effort and thus lack real-time and require observers to be there. Irrigation network[] leverages wireless sensor network technology to monitor water flow rate in each water channel and control water flow rate in real time. 
Water flow rate estimate techniques in closed pipes are of two types; intrusive sensors and nonintrusive sensors. One of the most popular intrusive water flow rate sensors is main water flow meter in houses and building for billing purpose. It provides accurate water flow rate and total amount of water consumed while guaranteeing long time reliability. It uses a mechanical turbine in a pipe that rotates proportional to water flow rate in the pipe. By counting number of pulses generated by the turbine, it calculates water flow rate and total amount of water consumed. While it is proven to be a reliable and robust measurement method to monitor water flow rate in a pipe, its installation cost is high as it requires us to cut a pipe and add the monitoring device unless they make the plumbing system from the initial construction. 
Ultrasonic based water flow meters[] are commercially available non-intrusive water flow rate measurement units that do not require to cut any pipe. Instead it uses one ultrasonic transmitter and receiver pair on a pipe and measure water flow rate induced doppler shift. Although it is less-intrusive, its cost is pretty expensive, more than \$1000 per a unit and requires careful installation, thus is for industrial purpose and pipe testing purpose only, therefore, it is infeasible to have every pipe equipped with this costly sensors.
Water flow induced vibration[Evans] is another way to infer water flow rate in a pipe. It could be inexpensive as it uses a simple accelerometer and processing of data is pretty simple. However, its calibration requires the actual water flow rate is a pipe and human intervention is also an issue. Our approach relies on this vibration based approach as it is less intrusive and much cheaper than ultrasonic based technique. But we also make use of pre-existing infrastructure that provides more precise water flow rate information that can help calibrate the vibration sensors on the fly. 
Infrastructure often provides many useful information as it has many different types of information which is often readily available and many of them are quiet accessible as they are designed to be monitored with some interfacing circuitry or many available techniques exist as monitoring of infrastructure were necessary for their maintenance purpose. [Fogarty][PowerLine] actively use this type of information to monitor activities in a house hold. [Powerline] monitors electrical noise in power-lines to infer status of appliances by monitoring a unique noise pattern of each of appliances. Since its purpose is to infer activities regarding electronic appliances, it does not provide the actual power consumption of appliances. In a similar line, Fogarty[] investigated a way to monitor plumbing system to infer water related activities while monitoring noise using microphones on many of pipes. Both of them has a great merit as it is less intrusive as they monitor somewhat hidden infrastructure and users may not be aware of their existence. However, complex calibration procedure requires a well trained technician to install and calibrate their systems, and this has to be addressed to make the system to be a mass deployable solution.
