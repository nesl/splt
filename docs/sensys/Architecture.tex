\section{ NAWMS Architecture}

To make the system mass deployable and to reduce cost, NAWMS exploits many available options. Especially, it extensively uses commercially available devices to infer water flow rate in a pipe in a nonintrusive way while leveraging pre-existing information from the infrastructure, namely the main water meters for billing purpose. 
The main three components of the system consist of : 
 Snooping of Existing Infrastructure; The main water meter provides accurate water flow rate as it is meant to be for billing purpose. It, however, lacks spatial granularity as it only needs to monitor water flow rate to the entire household. Spatially finer granularity is not necessary for the main water meter. 
 Wireless Sensor Nodes; it inherently has many types of sensing modalities. The system uses accelerometer to measure water flow rate relying on mechanical characteristics of water flow and pipe structure.
 Open Source Optimization Toolbox; calibration requires much time and human intervention. This paper propose several types of optimization problems that enable the system to calibrate its parameters and coefficients automatically. 

\subsection{ Snooping of Existing Infrastructure }
Available technologies for resource monitoring either provide information at coarse spatial granularity, like the water meters for an apartment. The problem is that they miss to show the consumption at a per individual level. 
However, they provide accurate sensor reading that helps calibrate sensors that monitor physically connected sensors as they are less precise as shown in section 3, but monitor partial information of the main water meter. As we have shown in section (), we make use of this relatively accurate information to maximize monitoring accuracy. In other words, NAWMS aims to aggregate both types of information, M(t) ad v(t), to hit the balance among these conflicting criteria. Since most households already have measurement units, e.g. water meters, NAWMS snoops the information from the main water meter.  
To get real-time information from the water meter, some technology can be used. [Flowmeter][][] is one way to get an accurate water flow rate of the main water pipe in a house as it provides pulse train depending on the actual water flow rate in a pipe. Most of wireless sensor nodes can get this information and provides real-time water flow rate to the place where it is needed. 

\subsection{ Nonintrusive Water Flow Rate using WSNs}

Intuition tells us that vibration of a pipe can be a metric to estimate water flow rate in a pipe. The sophisticated theory of operation is give in section ??. This section describes its hardware set-up to monitor water flow induced vibration in a pipe. 
We used MicaZs with MTS310[] data acquisition boards running SOS operating system[] to measure vibration on a pipe. The MTS310 sensor board includes several sensors such as microphone, accelerometer, magnetometer and photo-resistive light sensor. We only used one analog device ADXL202 accelerometer that is capable of measuring ? range of acceleration. To achieve better accuracy, one can use more sensitive vibration sensors. Since the problems ()() constrain the maximum flow rate of each pipe, the external vibration will be disregarded. (good part!). To reduce development cost, we took this as it achieves reasonable accuracy as the overall precision is benign and is for human reading purpose for now. 
Static SOS application samples acceleration 100Hz, after getting 50 samples, it locally calculates sample variance of acceleration and we consider this as measure of vibration[][]. As it processes the data locally and send the vibration information back to the aggregation every one second, its communication cost is low. It is also obvious the from the equation() the measure of vibration of this case is the sample variance of acceleration perpendicular to the pipe axis.

\subsection{Optimization Toolbox}
To solve the optimization problems given in section ??, we used CVX toolbox[][] that is an open source convex optimization tool and has python, c, and Matlab interfaces. Its efficiency and reliability is known to be good and it can solve most of convex optimization problem in polynomial time while guaranteeing robust solution that is also a global optimum. More complicated tool also can be used [Mosek] since they provide student license for free, but our optimization is well fit to the CVX thus we just used that one. 

