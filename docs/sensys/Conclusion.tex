\section{Conclusion}
We introduced less intrusive and auto calibrated individual pipe level water monitoring system that provides spatially finer grained water usage profile that helps people conserve water resource in a house hold. The solutions of the optimization problems yield calibration parameters and vibration propagation parameters that minimize the sum of estimation error. Our experimental results showed that the vibration based water flow rate estimation is feasible as it renders reasonably small error. 
We also envision that this information helps classify water related activities similar to [Fogarty] and further classification accuracy. It may also be used for detecting water leakage by determining if there�s excessive water flow happens on a pipe than the normal operation along a similar line of [Structure Monitoring]. 

\section{ Future work}
The limitation of the current system is that it requires us to put many sensors on every pipe of interest, that may increase hardware cost as the pipe system becomes complex and has many pipes. We will explore ways of estimating water flow rate using less amount of sensors considering pipe topology. 
With slight modification, we envision the proposed frame work can also be used for electrical energy consumption as it consists of the main power meter for billing purpose and some less intrusive sensors that need to be calibrated are available [leland]. 
In terms of applications, activity classification can be very interesting as water usage often requires human presence and the water related activities result in water consumption similar to [], but can provide more meaningful information for users in terms of natural resource consumption profile associated with activities thus help people improve their resource efficiency. 